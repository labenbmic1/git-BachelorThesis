%% begin: LaTeX-TikZ-Examples.tex
%% This is a collection of TikZ-images, but only for the purpose of debugging (and a few examples due to requests). Therefore the tikzset-options were not included into tumtikz.sty, because this are only "local" settings for this file and therefore not very "good"... . For the created images, see /images/tikz/.

\tikzset{%
	circuit declare symbol = ammeter,
	set ammeter graphic = {%
		circuit symbol lines, generic circle IEC, info = center:A, circuit symbol size = width	2.5 height 2.5
	},%
	circuit declare symbol = voltmeter,
	set voltmeter graphic = {%
		circuit symbol lines, generic circle IEC, info = center:V, circuit symbol size = width	2.5 height 2.5
	},%
	circuit declare symbol = ac source,%
	set ac source graphic = ac source IEC graphic,%
	ac source IEC graphic/.style={%
		transform shape,
		circuit symbol lines,
		circuit symbol size = width 2.5 height 2.5,
		shape = generic circle IEC,
		/pgf/generic circle IEC/before background=
		{
			\pgfpathmoveto{\pgfpoint{-0.8pt}{0pt}}
			\pgfpathsine{\pgfpoint{0.4pt}{0.4pt}}
			\pgfpathcosine{\pgfpoint{0.4pt}{-0.4pt}}
			\pgfpathsine{\pgfpoint{0.4pt}{-0.4pt}}
			\pgfpathcosine{\pgfpoint{0.4pt}{0.4pt}}
			\pgfusepathqstroke
		}
	}%
}%

\tikzset{%
	%% line width, draw, fill, minimum size, outer sep, inner sep, 
	%% huge circuit symbols, 
	%% aren't global here and can be changed after using the command
	every picture/.style={%
		w3, huge circuit symbols, circuit logic IEC, circuit ee IEC,%
		%	
	},%
	circ/.style={%
		w3, draw = black, fill = black, circle, minimum size = 1.5mm, outer sep = 0mm, inner sep = 0mm%
	},%
	ocirc/.style={%
		w3, draw = black, fill = white, circle, minimum size = 1.5mm, outer sep = 0mm,
		inner sep = 0mm%	
	}%
}%

\tikzset{external/export=true}
\setchemformula{tikz-external-disable=true}

\chapter{Auswertung von Daten}
\label{chap:Auswertung von Daten}
\section{Visualisierung mit TikZ}
\label{sec:Auswertung:Visualisierung mit TikZ}
\begin{figure}\centering
	\tikzsetnextfilename{Figure-1} 
	\begin{tikzpicture}[x=10mm,y=10mm]
		\draw[very thin,color=gray] (-0.1,-1.1) grid (9.9,3.9);
		\draw[->] (-0.2,0) -- (10.2,0) node[right] {$x$}; 
		\draw[->] (0,-1.2) -- (0,4.2) node[above] {$f(x)$};
		\draw[color=red]    plot[domain=0:4] (\x,\x)             node[right] {$f(x) =x$}; 
		\draw[color=blue]   plot[domain=0:9,samples=10] (\x,{sin(\x r)})    node[right] {$f(x) = \sin x$}; 
		\draw[color=black]   plot[domain=0:10,samples=100] (\x,{cos(\x r)})    node[right] {$f(x) = \cos x$}; 
		\draw[color=orange] plot[domain=0:4] (\x,{0.05*exp(\x)}) node[right] {$f(x) = \frac{1}{20} \mathrm e^x$};
	\end{tikzpicture}
	\caption{Plotting a function with TikZ}
	\label{fig:Auswertung:Visualisierung:Plotting a function with TikZ}
\end{figure}

\begin{figure}\centering
	\newcommand{\extrayticklist}{}%
	\let\extrayticklist=\empty%
	\makeatletter
	\foreach \n  in {0,1,2,3,4,5,6}{
		\foreach \m in {1,2,3,4}{
			\pgfmathparse{(exp(\n)-exp(\n-1))/5*\m + exp(\n-1)}%
			\ifx\empty\extrayticklist{} \protected@xdef\extrayticklist{\pgfmathresult}%
			\else \protected@xdef\extrayticklist{\extrayticklist,\pgfmathresult}%
			\fi
		}
	}
	\makeatother
	\pgfplotsset{lua backend=false}% ymode=log
	\tikzsetnextfilename{Figure-2} 
	\begin{tikzpicture}[x=10mm,y=10mm]
		\pgfmathsetmacro{\expone}{exp(1)}
		\begin{axis}[%
			xmode=linear,%
			ymode=log,%
			axis x line = bottom,%
			axis y line = left,%
			axis on top,%
			height=0.5\textwidth,
			width=0.7\textwidth,
			log basis ticks={y},
			xtick = {0,5,...,35},
			minor x tick num=4, 
			xtick style = {orange},
			xtick align = outside,
			grid = major,%
			major grid style = {solid, green!50!black},
			log basis y={\expone},
			log number format basis/.code 2 args={
				$e^{\pgfmathprintnumber{#2}}$},
			extra y ticks/.expanded ={\extrayticklist},
			every extra y tick/.append style = {% 
				grid = major,%
				major grid style = {dotted, orange!50!black},
				tick style = {blue},%
				tick align = outside% 
			},
			extra y tick labels={},%
			extra x tick style = {% 
				grid = none,%
				tick style = {orange},%
				tick align = outside%
			},%
			xlabel={$t$ in $s$},
			ylabel={$\varphi$ in $^{\circ}$},
			xmin=0, xmax=36, ymin=1, ymax={exp(5.2)},
			mark size=1mm,
			legend pos=outer north east, legend cell align=left,
			legend style ={% 
				draw=black, 
				fill=white,},%
			]
			\addplot[domain=0:5, samples=10, dotted, mark=triangle*, mark options={ fill=white}] {exp(x)};
			\addlegendentry{$f(x)=e^{x}$}
			\addplot[domain=5:10, samples=10, mark=star, mark options={fill}, dashed, color=blue] {exp(x-5)/2};
			\addlegendentry{$f(x)=\frac{e^{x-5}}{2}$}
			\addplot[domain=0:35, densely dashdotted, samples=10, color=red, mark=diamond*, mark options={fill=gray}] {x} node[pos=0.9, anchor=north] {$f(x)=x$};
			\addlegendentry{$f(x)=x$}
		\end{axis}
	\end{tikzpicture}
	\pgfplotsset{lua backend=false}% ymode=log
	\caption{Plotting a function with axis-environment}
	\label{fig:Auswertung:Visualisierung::Plotting a function with axis-environment}
\end{figure}

\tikzsetnextfilename{Figure-3} 
\begin{tikzpicture}[x=10mm,y=10mm]
	\begin{axis}[xmode=linear, ymode=linear, axis x line=center, axis y line=center, TUM style 1, xmin=0, xmax=1, ymin=0, ymax=1, grid=both, 
		enlarge x limits={rel=0.05,upper},
		enlarge y limits={rel=0.05,upper},
		xtick distance=0.2, minor x tick num=5,
		ytick distance=0.2, minor y tick num=5, 
		samples=50, domain=0:1]
		\addplot {x};
		\addplot {x/1.2};
		\addplot {x/1.4};
		\addplot {x/1.6};
		\addplot {x/1.8};
		\addplot {x/2};
		\addplot {x/2.5};
		\addplot {x/3};
		\addplot {x/5};
		\addplot {x/10};
		\addplot {x/20};
	\end{axis}
\end{tikzpicture}

\begin{figure}
	\tikzsetnextfilename{Figure-4} 
	\begin{tikzpicture}
		\begin{axis}[xmode=linear, ymode=linear, axis y line=center,
			axis x line=center, TUM style 1,
			width=\textwidth,
			height=0.3\textwidth, 
			tick align=center, hide obscured x ticks=true,
			ymin=-80, ymax=80, xmin=0, xmax=0.045,
			enlarge y limits={abs=19},
			enlarge x limits={abs=0.0049,upper},
			xtick distance=0.005, minor x tick num=4,
			ytick distance=40, minor y tick num=3, 
			ymajorgrids, yminorgrids,
			ylabel={$u(t)$ in $\mathrm{V}$}, xlabel={$t$ in $\mathrm{s}$}, 
			every axis x label/.style={at={(axis description cs:0.95,0.5)}, anchor=south},%
			every axis y label/.style={at={(axis description cs:0,1)}, anchor=south},
			scaled x ticks={base 10:3},
			x tick label style={/pgf/number format/.cd,fixed,precision=1,/tikz/.cd},%
			legend cell align=left,
			legend style ={% 
				at={(1,1)}, anchor=south east,
				draw=black, 
				fill=white,}%
			]%
			\addplot [color=RedViolet] table[skip first n=1,x index=2,y index=3, col sep = comma] {files/oszi/help/CH1.csv};
			\addlegendentry{$u_{1}(t)$}
			\addplot [color=ForestGreen] table[skip first n=1,x index=2,y index=3, col sep = comma] {files/oszi/help/CH2.csv};
			\addlegendentry{$u_{2}(t)$}
		\end{axis}
	\end{tikzpicture}
	\caption{Zeitverläufe der Messgrößen bei $ \alpha = 0 \, ^{\mathrm{ \circ}} $ (B2C mit Energiespeicher)}
	\label{fig:Auswertung:Visualisierung:Zeitverlaeufe der Messgroessen bei alpha=0 (B2C mit Energiespeicher)}
\end{figure}

\begin{figure}
	\centering
	\tikzsetnextfilename{Figure-5} 
	\begin{tikzpicture}[x=1mm, y=1mm]%
		%% ** 1. \coordinate[⟨options⟩] (⟨name⟩) at (⟨coordinate⟩);
		%% same as \path coordinate 
		%% options ... i. e. [label = left:$A$]
		\coordinate (origin) at (0,0);%
		\def\localDistanceLength{0.9\textwidth}%
		\def\localDistanceHeight{40}%
		\def\globalGroundDistance{3.333}%
		\coordinate (lo) at (-\localDistanceLength/2,\localDistanceHeight/2);%
		\coordinate (ru) at (\localDistanceLength/2,-\localDistanceHeight/2);%
		\coordinate (lu) at (lo |- ru);%
		\coordinate (ro) at (lo -| ru);%
		\coordinate (Ao) at ($(lo)!1/5!(ro)$);%
		\coordinate (Bo) at ($(lo)!2/5!(ro)$);%
		\coordinate (Co) at ($(lo)!3/5!(ro)$);%
		\coordinate (Do) at ($(lo)!4/5!(ro)$);%
		\coordinate (Au) at (Ao |- ru);%
		\coordinate (Bu) at (Bo |- ru);%
		\coordinate (Cu) at (Co |- ru);%
		\coordinate (Du) at (Do |- ru);%
		%
		%% ** 2. \draw, \node, etc.
		\draw (lo) to [resistor = {info' = {$R_{1} = \SI{10}{\ohm}$}}, small circuit symbols] (Ao);%
		\draw (Ao) to [resistor = {info = {$R_{2}$}}] (Bo);%
		\draw (Bo) to [capacitor = {info = {$C_{1}$}}] (Co);%
		\draw (Co) to [inductor = {color = blue}] (Do);%
		\draw (Bu) -- (Au);%
		\draw (Bu) to [ammeter = {color = red}, small circuit symbols, color = blue] (Cu);
		\draw (Cu) to [voltmeter] (Du);
		\draw (lo) to [ac source] (lu);
		
		\node[rectangle, minimum size = 15mm, draw = black, align = center] (rec1) at (0,0) {OSZ\\ X};
		\draw (Bo) |- ([yshift = 2.5mm]rec1.west);
		\draw (Bu) |- ([yshift = -2.5mm]rec1.west);
		
		\draw (Bu) -- ([shift={(0,-\globalGroundDistance)}]Bu) node[ground, point down, anchor=west] (g1) {};
		
		\foreach \x/\y in {oc1/lo,oc2/lu,oc3/ro,oc4/ru}{
			\node[ocirc] (\x) at (\y) {};
		}
		\foreach \x/\y in {c1/Ao,c2/Au,c3/Bo,c4/Bu,c5/Co,c6/Cu,c7/Do,c8/Du}{
			\node[circ] (\x) at (\y) {};
		}
		
		%% ** 3. scope for layers
		\begin{scope}[on background layer]
			\node (nothing) {};
		\end{scope}
	\end{tikzpicture}
	\caption{Schaltbild}
	\label{fig:Auswertung:Visualisierung:Schaltbild}
\end{figure}

\tikzsetnextfilename{Figure-6} 
\begin{tikzpicture}[baseline]
	\begin{axis}[xmode=linear, ymode=linear, axis y line=left,
		axis x line=bottom, TUM style 1,
		width=0.5\textwidth,
		height=0.5\textwidth, 
		tick align=outside,
		ymin=0, ymax=10, xmin=0, xmax=200,
		enlarge y limits={abs=0.9,upper},
		enlarge x limits={abs=19,upper},
		xtick distance=50, minor x tick num=4,
		ytick distance=2, minor y tick num=1, 
		xlabel={I/mA},
		ylabel={R/$\Omega$}, grid=both]%
		%
		\addplot+[only marks, error bars/.cd, y dir = both, y explicit, x dir = both, x explicit] table[y error = dy, x error = dx] {files/help/test.dat};%
		%
	\end{axis}
\end{tikzpicture}%

\newlength{\tikZlengthV}
\newlength{\tikZlengthH}

\begin{figure}
	\centering
	\tikzsetnextfilename{Figure-7} 
	\begin{tikzpicture}[x=1mm, y=1mm]
		\coordinate (origin) at (0,0);%
		\def\localDistanceLength{0.15\textwidth}%
		\def\localDistanceHeight{0.15\textwidth}%	
		% 1.
		\coordinate (lo) at (-\localDistanceLength/2,\localDistanceHeight/2);%
		\coordinate (ru) at (\localDistanceLength/2,-\localDistanceHeight/2);%
		\coordinate (lu) at (lo |- ru);%
		\coordinate (ro) at (lo -| ru);%
		
		\begin{scope}[shift={(lu)}]
			\clip (lu) rectangle (ro);
			\foreach \p in{1,...,50}{%
				\node[circle,inner sep=0pt, outer sep=0pt,fill=TUMcolorII, minimum width=1mm] at (1*\localDistanceLength*rnd,1*\localDistanceHeight*rnd) {};%
			}%
		\end{scope}
		
		\draw[] (lu) rectangle (ro);
		\node[circle,inner sep=0pt, outer sep=0pt,draw = black, minimum width=\localDistanceLength] at (origin) {};%
		
		\coordinate (lol) at ([shift={(-0.15*\localDistanceLength,0)}]lo);
		\coordinate (lor) at ([shift={(0.08*\localDistanceLength,0)}]lol);
		\coordinate (lom) at ($(lol)!0.5!(lor)$);
		\coordinate (lul) at (lol|-lu);
		\coordinate (lum) at (lom|-lu);
		\coordinate (lur) at (lor|-lu);
		\draw (lol) -- (lor) (lul) -- (lur);
		\draw[<->] (lom) -- node[label=left:{$1$}]{} (lum);
		\coordinate (lol) at ([shift={(0,-0.15*\localDistanceLength)}]ru);
		\coordinate (lor) at ([shift={(0,0.08*\localDistanceLength)}]lol);
		\coordinate (lom) at ($(lol)!0.5!(lor)$);
		\coordinate (lul) at (lol-|lu);
		\coordinate (lum) at (lom-|lu);
		\coordinate (lur) at (lor-|lu);
		\draw (lol) -- (lor) (lul) -- (lur);
		\draw[<->] (lom) -- node[label=below:{$1$}]{} (lum);
	\end{tikzpicture}
	\caption{Monte Carlo\ldots{}}
	\label{fig:Auswertung:Visualisierung:MonteCarlo}
\end{figure}

\begin{figure}
	\centering
	\tikzsetnextfilename{Figure-8} 
	\begin{tikzpicture}[x=7mm, y=4mm]
		\def\a{1.7}
		\def\b{5.7}
		\def\c{3.7}
		\def\L{0.5} % width of interval
		
		\pgfmathsetmacro{\Va}{2*sin(\a r+1)+4} \pgfmathresult
		\pgfmathsetmacro{\Vb}{2*sin(\b r+1)+4} \pgfmathresult
		\pgfmathsetmacro{\Vc}{2*sin(\c r+1)+4} \pgfmathresult
		
		\draw[->,thick] (-0.4,0) -- (8,0) coordinate (x axis) node[below] {$x$};
		\draw[->,thick] (0,-0.7) -- (0,7) coordinate (y axis) node[left] {$y$};
		\foreach \f in {1.7,2.2,...,6.2} {\pgfmathparse{2*sin(\f r+1)+4} \pgfmathresult
			\draw[fill=\TUMcolor{16}!20] (\f-\L/2,\pgfmathresult |- x axis) -- (\f-\L/2,\pgfmathresult) -- (\f+\L/2,\pgfmathresult) -- (\f+\L/2,\pgfmathresult |- x axis) -- cycle;}
		\node at (\a-\L/2,-5pt) {{$a$}};
		\node at (\b+\L/2+\L,-5pt) {{$b$}};
		\draw[color=TUMBlue,thick,smooth,samples=100,domain=1:6.6] plot(\x,{2*sin(\x r+1)+4});
	\end{tikzpicture}
	\caption{Integration\ldots{}}
	\label{fig:Auswertung:Visualisierung:Integration}
\end{figure}


\begin{figure}
	\centering
	\tikzsetnextfilename{Figure-9} 
	\begin{tikzpicture}[baseline]
		\begin{axis}[xmode=linear, ymode=linear, 
			major grid style={solid, black, w2},%										%%
			minor grid style={dotted, black, w1},%										%%
			inner axis line style={solid, black, w3},%									%%
			width=0.95\textwidth,
			height=0.6\textwidth, 
			tick align=inside,
			axis line style={-},%														%%
			ymin=0, ymax=1.1, xmin=-17, xmax=17,
			xtick distance=5, minor x tick num=4,
			ytick distance=0.2, minor y tick num=1, 
			xlabel={Transversalimpuls in $\si{\milli\radian}\cdot m_\mathrm{e} c$},
			ylabel={Koinzidenz-Zählrate in a.u.}, grid=both,
			legend cell align={left},
			legend style={%
				at={(axis cs:17,1.1)},%
				anchor=north east%
			}%
			]%
			%
			\addplot+[only marks, mark=x, color=TUMcolorII, TUMcolorII, 
			error bars/.cd, x dir=none, y dir=both, y explicit%
			] table[col sep=comma, y error index=2]  {files/help/test-tikz-2.txt};%
			\addplot+[mark=none, color=TUMBlue, TUMBlue] table[col sep=comma]  {files/help/test-tikz-1.txt};%
			\addplot+[color=TUMBlue, TUMBlue, dashed, mark=none, domain=-10:10] {0.371634 + 5.67826*10^(-9)*x - 0.0132079*x*x};% python...
			\legend{normierte koinzidente Ereignisse, gesamter Fit, parabolischer Anteil des Fits}
			%
		\end{axis}
	\end{tikzpicture}%
	\caption{Data and fit function :P (but please do this in, for example, python and only use \texttt{includegraphics})\ldots{}}
	\label{fig:Auswertung:Visualisierung:Data}
\end{figure}


\tikzset{%
	block/.style n args={2}%
	{rectangle,minimum width=15mm, minimum height=12.5mm,align=center, w3,%
		draw=#1, fill=#2, drop shadow},%
	block/.default={{red}{TUMBlue!24!white}},%
	block-regular/.style={block={#1}{#1!24!white}},%
	block-regular/.default={{TUMcolorII}}%
}%
\begin{figure}
	\centering%
	\tikzsetnextfilename{Figure-10}%
	\begin{tikzpicture}[x=1mm,y=1mm]
		\coordinate (ref) at (0,0);
		\coordinate (probe) at (ref);
		\coordinate (detektor1) at ([shift={(-0.2\textwidth,0)}]ref);
		\gettikzxy{(detektor1)}{\ax}{\ay}
		\coordinate (detektor2) at (-\ax,\ay);
		\node (probe) [block-regular={TUMBlue}, circle] at (ref) {Probe};
		\node (detektor1) [block-regular] at (detektor1) {Detektor 1};
		\node (detektor2) [block-regular] at (detektor2) {Detektor 2};
		
		\gettikzxy{(detektor1)}{\ax}{\ay}
		\coordinate (SA1) at ([shift={(0,-10mm)}]2*\ax,0);
		\gettikzxy{(detektor2)}{\ax}{\ay}
		\coordinate (SA2) at ([shift={(0,-10mm)}]2*\ax,0);
		\node (SA1) [block-regular] at (SA1) {Spectroscopy\\ Amplifier};
		\node (SA2) [block-regular] at (SA2) {Spectroscopy\\ Amplifier};
		
		\gettikzxy{(SA1)}{\ax}{\ay}
		\coordinate (SCA1) at ([shift={(0,-20mm)}]\ax,\ay);
		\gettikzxy{(SA2)}{\ax}{\ay}
		\coordinate (SCA2) at ([shift={(0,-20mm)}]\ax,\ay);
		\node (SCA1) [block-regular] at (SCA1) {Single Channel\\ Analyser};
		\node (SCA2) [block-regular] at (SCA2) {Single Channel\\ Analyser};
		
		\gettikzxy{(detektor1)}{\ax}{\ay}
		\gettikzxy{(SCA1)}{\bx}{\by}
		\coordinate (verzoegerung) at ([shift={(0,-10mm)}]\ax,\by);
		\node (verzoegerung) [block-regular] at (verzoegerung) {Signal-\\ verzögerung};
		
		\gettikzxy{(probe)}{\ax}{\ay}
		\gettikzxy{(verzoegerung)}{\bx}{\by}
		\coordinate (koinzidenz) at (\ax,\by);
		\node (koinzidenz) [block-regular] at (koinzidenz) {Koinzidenz-\\ einheit};
		
		\gettikzxy{(koinzidenz)}{\ax}{\ay}
		\coordinate (zaehler) at ([shift={(0,-15mm)}]\ax,\ay);
		\node (zaehler) [block-regular, minimum height=7.5mm] at (zaehler) {Zähler};
		\gettikzxy{(zaehler)}{\ax}{\ay}
		\coordinate (PC) at ([shift={(0,-15mm)}]\ax,\ay);
		\node (PC) [block-regular] at (PC) {PC};
		
		\draw[-] (detektor1)-|(SA1)--(SCA1)|-(verzoegerung)--(koinzidenz);
		\draw[-] (detektor2)-|(SA2)--(SCA2)|-(koinzidenz);
		\draw[-] (koinzidenz)--(zaehler);
		\draw[-] (zaehler)--(PC);
		\draw[-] (SCA2)|-(zaehler);
		\gettikzxy{(SCA2)}{\ax}{\ay}
		\gettikzxy{(koinzidenz)}{\bx}{\by}
		\node[circle,fill=black,minimum width=1mm,draw=black,inner sep=0pt,outer sep=0pt] at (\ax,\by) {};
		
		\gettikzxy{(SA2)}{\ax}{\ay}
		\gettikzxy{(SCA2)}{\bx}{\by}
		\gettikzxy{(detektor2)}{\cx}{\cy}
		\coordinate (schritt) at (\cx,\ay/2+\by/2);
		\node (schritt) [block-regular, circle] at (schritt) {Schritt-\\ motor};
		\draw[-] (PC)-|(schritt);
		
		\begin{scope}[on background layer]
			\coordinate (rec) at (\cx,\ay/2);
			\node [rectangle, draw=black, minimum width=3mm, minimum height = 30mm] at (rec) {};
		\end{scope}
		
		\gettikzxy{(detektor1)}{\ax}{\ay}
		\gettikzxy{(SA1)}{\bx}{\by}
		\coordinate (HV1) at (1.5*\ax,-\by);
		\node (HV1) [block-regular, minimum width=7.5mm, minimum height=7.5mm] at (HV1) {HV};
		\draw[-] ([shift={(0,2.5)}]detektor1)-|(HV1);
		\gettikzxy{(HV1)}{\ax}{\ay}
		\coordinate (HV2) at (-\ax,\ay);
		\node (HV2) [block-regular, minimum width=7.5mm, minimum height=7.5mm] at (HV2) {HV};
		\draw[-] ([shift={(0,2.5)}]detektor2)-|(HV2);
	\end{tikzpicture}
	\caption{Schematischer Aufbau und Schaltplan des Versuchs.}%
	\label{fig:Auswertung:Visualisierung:Schematischer Aufbau und Schaltplan des Versuchs}%
\end{figure}




\tikzset{external/export=false}
%% end: LaTeX-TikZ-Examples.tex