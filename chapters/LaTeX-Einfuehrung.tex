%% begin: LaTeX-Einfeuhrung.tex
\chapter{Einleitung}
\label{chap:Einleitung}
\section{Dokumentaufbau}
\label{sec:Einleitung:Dokumentaufbau}
\lipsum[1-2]
\section{Die Geschichte von \texttt{KOMA-Script}}
\label{sec:Einleitung:Geschichte}
\lipsum[1-2]
\section{Installation}
\label{sec:Einleitung:Installation}
\lipsum[1-2]
\section{Fehlermeldungen, Fragen, Probleme}
\label{sec:Einleitung:Fehlermeldung, Frage, Probleme}
In Kap. \ref{chap:Einleitung} wird \ldots{}.\par%
Das ist ein Test.\par\smallskip
Das war ein \verb|\parsmallskip|.
\lipsum[1-2]
\lipsum[1-3]
\lipsum[1-4]


\part[Einführung in \LaTeX{}, KOMA-Script für Autoren\texorpdfstring{, \cite{LabenbacherTeX}}{}]{Einführung in \LaTeX{}, KOMA-Script für Autoren\texorpdfstring{, \cite{LabenbacherTeX}, \protect\footnote{In \texttt{part}.}}{}}%
\label{part:Einfuehrung in LaTeX, KOMA-Script fuer Autoren, scrbook}%
\chapter[Satzspiegelberechnung mit \texttt{typearea.sty}]{Satzspiegelberechnung mit \texttt{typearea.sty}\texorpdfstring{, \protect\footnotemark{}}{}}\footnotetext{In \texttt{chapter}.}%
\label{chap:Einfuehrung:Satzspiegelberechnung}%
% mbox, only due to underlineing :P
\section[Grundlagen der Satzspiegelkonstruktion]{Grundlagen der Satzspiegelkonstruktion\texorpdfstring{, \cite{LabenbacherTeX}, \ref{sec:Einfuehrung:Hauptklassen:Abgrenzung}, \eqref{equ:Mathematik:E = mc2}}{}\texorpdfstring{, \protect\mbox{\footnotemark}}{}}\footnotetext{In \texttt{section}.}%
\label{sec:Einfuehrung:Satzspiegelberechnung:Grundlagen}%
\subsection[Weiterentwicklung]{Weiterentwicklung\texorpdfstring{, \protect\mbox{\footnotemark}}{}}\footnotetext{In \texttt{subsection}.}%
\label{subsec:Einfuehrung:Satzspiegelberechnung:Grundlagen:Weiterentwicklung}%
\subsubsection[Endstation\texorpdfstring{ (nicht erlaubt in Büchern eigentlich ab incl. hier!)}{}]{Endstation\texorpdfstring{ (nicht erlaubt in Büchern eigentlich ab incl. hier!)}{}\texorpdfstring{, \protect\footnotemark}{}}\footnotetext{In \texttt{subsubsection}.}%
\label{subsubsec:Einfuehrung:Satzspiegelberechnung:Grundlagen:Weiterentwicklung:Endstation}%
\paragraph[Abschnitt]{Abschnitt\texorpdfstring{, \protect\footnote{In \texttt{paragraph}.}}{}}%
\label{para:Einfuehrung:Satzspiegelberechnung:Grundlagen:Weiterentwicklung:Endstation:Abschnitt}%
\minisec{Miniüberschrift ohne Referenzierung}
\lipsum[1-1]%
\section{Satzspiegelkonstruktion durch Teilung}%
\label{sec:Einfuehrung:Satzspiegelberechnung:Satzspiegelkonstruktion Teilung}%
\lipsum[1-1]
\section{Satzspiegelkonstruktion durch Kreisschlagen}
\label{sec:Einfuehrung:Satzspiegelberechnung:Satzspiegelkonstruktion Kreisschlagen}
\lipsum[1-1]
\section{Frühe oder späte Optionenwahl}
\label{sec:Einfuehrung:Satzspiegelberechnung:Optionenwahl}
\lipsum[1-1]
\section{Kompatibilität zu früheren Versionen von \texttt{KOMA-Script}}
\label{sec:Einfuehrung:Satzspiegelberechnung:Kompatibilitaet}
\lipsum[1-1]
\section{Einstellung des Satzspiegels und der Seitenaufteilung}
\label{sec:Einfuehrung:Satzspiegelberechnung:Einstellung Satzspiegel}
\lipsum[1-1]
\section{Einstellung des Papierformats}
\label{sec:Einfuehrung:Satzspiegelberechnung:Einstellung Papierformat}
\lipsum[1-1]
\section{Tipps}
\label{sec:Einfuehrung:Satzspiegelberechnung:Tipps}
\lipsum[1-1]

\chapter{Die Hauptklassen \texttt{scrbook, scrreprt, scrartcl}}
\label{chap:Einfuehrung:Hauptklassen}
\section{Frühe oder späte Optionenwahl}
\label{sec:Einfuehrung:Hauptklassen:Optionenwahl}
\lipsum[1-1]
\section{Kompatibilität zu früheren Versionen von \texttt{KOMA-Script}}
\label{sec:Einfuehrung:Hauptklassen:Kompatibilitaet}
\lipsum[1-1]
\section{Entwurfsmodus}
\label{sec:Einfuehrung:Hauptklassen:Entwurfsmodus}
\lipsum[1-1]
\section{Seitenaufteilung}
\label{sec:Einfuehrung:Hauptklassen:Seitenaufteilung}
\lipsum[1-1]
\section{Wahl der Schriftgröße für das Dokument}
\label{sec:Einfuehrung:Hauptklassen:Wahl Schriftgroesse}
\lipsum[1-1]
\section{Textauszeichnungen}
\label{sec:Einfuehrung:Hauptklassen:Textauszeichnungen}
\lipsum[1-1]
\section{Dokumenttitel}
\label{sec:Einfuehrung:Hauptklassen:Dokumenttitel}
\lipsum[1-1]
\section{Zusammenfassung}
\label{sec:Einfuehrung:Hauptklassen:Zusammenfassung}
\lipsum[1-1]
\section{Inhaltsverzeichnis}
\label{sec:Einfuehrung:Hauptklassen:Inhaltsverzeichnis}
\lipsum[1-1]
\section{Absatzauszeichnung}
\label{sec:Einfuehrung:Hauptklassen:Absatzauszeichnung}
\lipsum[1-1]
\section{Erkennung von rechten und linken Seiten}
\label{sec:Einfuehrung:Hauptklassen:Erkennung rechts links}
\lipsum[1-1]
\section{Kopf und Fuß bei vordefinierten Seitenstilen}
\label{sec:Einfuehrung:Hauptklassen:Kopf und Fuss}
\lipsum[1-1]
\section{Vakatseiten}
\label{sec:Einfuehrung:Hauptklassen:Vakatseiten}
\lipsum[1-1]
\section{Fußnoten}
\label{sec:Einfuehrung:Hauptklassen:Fussnoten}
\lipsum[1-1]
\section{Abgrenzung}
\label{sec:Einfuehrung:Hauptklassen:Abgrenzung}
\lipsum[1-1]
\section{Gliederung}
\label{sec:Einfuehrung:Hauptklassen:Gliederung}
\lipsum[1-1]
\section{Schlauer Spruch}
\label{sec:Einfuehrung:Hauptklassen:Schlauer Spruch}
\lipsum[1-1]
\section{Listen}
\label{sec:Einfuehrung:Hauptklassen:Listen}
\lipsum[1-1]
\section{Mathematik}
\label{sec:Einfuehrung:Hauptklassen:Mathematik}
\lipsum[1-1]
\section{Gleitumgebungen für Tabellen und Abbildungen}
\label{sec:Einfuehrung:Hauptklassen:Gleitumgebungen}
\lipsum[1-1]
\section{Randnotizen}
\label{sec:Einfuehrung:Hauptklassen:Randnotizen}
\lipsum[1-1]
\section{Anhang}
\label{sec:Einfuehrung:Hauptklassen:Anhang}
\lipsum[1-1]
\section{Literaturverzeichnis}
\label{sec:Einfuehrung:Hauptklassen:Literaturverzeichnis}
\lipsum[1-1]
\section{Stichwortverzeichnis}
\label{sec:Einfuehrung:Hauptklassen:Stichwortverzeichnis}
\lipsum[1-1]

%% richtiges Einfuegen von pdf's
\clearpage\newgeometry{top=0pt}% remove top margin,									%%
\includepdf[pages=-,%																%%
	addtotoc={1,chapter,1,{title},{label}}]%										%%
	{files/help/help.pdf}%															%%
\clearpage\TUMStandardAreaMain%														%%

\part{Gleitumgebungen in \texttt{scrbook}\texorpdfstring{ siehe Abb.~\ref{fig:Gleitumgebungen:Grafik mit subcaptionbox ohne captionsetup (Standard: komafont)}}{}}
\label{part:Gleitumgebungen in scrbook}
\chapter{Abbildungen mit/ohne \texttt{figure}-Umgebung\texorpdfstring{, \cite{LabenbacherTeX}}{}}
\label{chap:Gleitumgebungen:Abbildungen figure-Umgebung}

\begin{figure}%
	\captionsetup{justification=raggedleft, textfont={small,color={orange}}, labelfont={Large,color={\TUMcolor{2}},bf,it}}%
	\includegraphics[scale=0.5,angle=45]{help}%
	\caption{Grafik mit \TUMstyle{1}{\LaTeX{}}-captionsetup in der \TUMstyle{1}{\LaTeX{}}-figure-Umgebung, siehe \ref{fig:Gleitumgebungen:Grafik mit captionsetup in der figure-Umgebung}}%
	\label{fig:Gleitumgebungen:Grafik mit captionsetup in der figure-Umgebung}%
\end{figure}%
%
{\captionsetup{type=figure,justification=centering}%
	\includegraphics[scale=0.5]{help}%
	\caption{Grafik mit \TUMstyle{1}{\LaTeX{}}-captionsetup- ohne der \TUMstyle{1}{\LaTeX{}}-center-Umgebung, siehe \ref{fig:Gleitumgebungen:Grafik mit captionsetup ohne der figure-Umgebung}}%
	\label{fig:Gleitumgebungen:Grafik mit captionsetup ohne der figure-Umgebung}%
}%
%
\begin{figure}%
\captionsetup{justification=centering}%
\subcaptionbox{A\label{subfig:Gleitumgebungen:mit subcaptionbox:A}}{\includegraphics[]{help.png}}%
\hfill%
\subcaptionbox{B\label{subfig:Gleitumgebungen:mit subcaptionbox:B}}{\includegraphics[]{help.png}}%
\hfill%
\subcaptionbox{C\label{subfig:Gleitumgebungen:mit subcaptionbox:C}}{\includegraphics[]{help.png}}%
\hfill%
\subcaptionbox{D, \cite{LabenbacherTeX}\label{subfig:Gleitumgebungen:mit subcaptionbox:D}}{\includegraphics[angle=-45]{help.png}}%
\caption{Grafik mit \TUMstyle{1}{\LaTeX{}}-subcaptionbox ohne \TUMstyle{1}{\LaTeX{}}-captionsetup (\texttt{komafont}, \cite{LabenbacherTeX})}%
\label{fig:Gleitumgebungen:Grafik mit subcaptionbox ohne captionsetup (Standard: komafont)}%
\end{figure}%
\newcounter{TUMcthelp}%
\setcounter{TUMcthelp}{1}%
\whiledo{\value{TUMcthelp} < 140}% 148
{%
	\begin{figure}%
		\includegraphics[scale=0.2, angle=\theTUMcthelp]{help}%
		\caption{Grafik \theTUMcthelp{}}% ohne label
	\end{figure}%
	\stepcounter{TUMcthelp}%
}%
\lipsum[1-1]
\begin{figure}%
	\includegraphics[scale=0.5]{help}%
	\caption{Test einer langen, langen, langen, langen, langen, langen, langen, langen, langen, langen, langen, langen, langen, langen, langen, langen Unterschrift }%
\end{figure}%
\lipsum[1-1]
\begin{figure}%
	\begingroup%
	\KOMAoption{captions}{centeredbeside}%
	\begin{captionbeside}[Titel des Bildes im lof (optional)]{Titel des Bildes mittig}[i][\linewidth]%
		% r, l, i, o (in, out, bei rechts/links seitigen Dokumenten moeglich)
		\includegraphics[width=0.2\textwidth]{help}
	\end{captionbeside}\label{fig:Gleitumgebungen:Titel des Bildes mittig}%
	\endgroup%
\end{figure}
\lipsum[1-1]%
\begin{figure}
	\begingroup%
	\KOMAoption{captions}{topbeside}% ( = bottombeside + \raisebox{}{})
	\begin{captionbeside}[Titel des Bildes im lof (optional)]{Titel des Bildes oben}[i][\linewidth]%
		% r, l, i, o (in, out, bei rechts/links seitigen Dokumenten moeglich)
		\raisebox{\dimexpr\baselineskip-\totalheight\relax}{%
		\includegraphics[width=0.2\textwidth]{help}}%
	\end{captionbeside}\label{fig:Gleitumgebungen:Titel des Bildes oben}%
	\endgroup%
\end{figure}
\lipsum[1-1]%
\begin{figure}
	\begin{captionbeside}[Titel des Bildes im lof (optional)]{Titel des Bildes unten (standard)}[i][\linewidth]%
		% r, l, i, o (in, out, bei rechts/links seitigen Dokumenten moeglich)
		\includegraphics[width=0.2\textwidth]{help}%
	\end{captionbeside}\label{fig:Gleitumgebungen:Titel des Bildes unten}%
\end{figure}
\lipsum[1-1]%


\chapter{Tabellarische Darstellung mit/ohne \texttt{table}-Umgebung}
\label{chap:Tabellen}
\begin{table}
	\caption{Tabelle mit \TUMstyle{1}{\LaTeX{}}-subtable ohne \TUMstyle{1}{\LaTeX{}}-captionsetup (\texttt{komafont}, \cite{LabenbacherTeX})}
	\label{tab:Tabellen:Tabelle mit latex-subtable ohne latex-captionsetup (Standard: komafont)}
	\begin{subtable}[b]{\linewidth}
		\caption{Subtable one}
		\label{subtab:Tabellen:mit latex-subtable:one}
	\begin{tabular}{|r|r|}
		tab & tab2
	\end{tabular}
	\end{subtable}\\
	\begin{subtable}[b]{\linewidth}\centering
		\caption{Subtable two, \cite{LabenbacherTeX}, \url{https://github.com/labenbmic1},  \ref{sec:Einfuehrung:Hauptklassen:Entwurfsmodus}}
		\label{subtab:Tabellen:mit latex-subtable:two}
	\begin{tabular}{|r|r|}
		tab & tab2
	\end{tabular}
	\end{subtable}
\end{table}

\begin{longtable}{C{0.25\textwidth}Z{p}{0.2\textwidth}{\raggedleft}l}%
	\caption{A long table}\label{tab:Tabellen:A long table}\\%
	\toprule	
	f-head, \cite{LabenbacherTeX}, \ref{chap:Einleitung} & f-head\footnotemark & f-head \\
	\midrule
	\endfirsthead
	\toprule
	head, \ref{fig:Gleitumgebungen:Grafik mit captionsetup ohne der figure-Umgebung} & head & head, \url{https://github.com/labenbmic1} \\
	\midrule
	\endhead
	\midrule
	foot, \autoref{tab:Tabellen:A long table} & foot & foot \\
	\bottomrule
	\endfoot
	\midrule
	l-foot, \cite{LabenbacherTeX} & l-foot, \ref{subfig:Gleitumgebungen:mit subcaptionbox:C} & l-foot \\
	\bottomrule
	\endlastfoot
	%% =========================== %%
	\footnotetext{A footnote in the tableentry-firsthead of longtable.\label{footnote:Tabellen:A footnote in the tableentry-firsthead}} a&b\footnote{A footnote in the tableentry-body of longtable.\label{footnote:Tabellen:A footnote in the tableentry-body}} &c\\ a&b&c\\ a&b&c\\ a&b&c\\ a&b&c\\ a&b&c\\ a&b&c\\ a&b&c\\ a&b&c\\ a&b&c\\ a&b&c\\ a&b&c\\ a&b&c\\ a&b&c\\ a&b&c\\ a&b&c\\ a&b&c\\ a&b&c\\ a&b&c\\ a&b&c\\ a&b&c\\ a&b&c\\ a&b&c\\ a&b&c\\ a&b&c\\a&b&c\\ a&b&c\\ a&b&c\\ a&b&c\\a&b&c\\ a&b&c\\ a&b&c\\ a&b&c\\a&b&c\\ a&b&c\\ a&b&c\\ a&b&c\\
\end{longtable}

Nun eine Messwerttabelle von Excel mit \verb|\input|.\\
%% Table generated by Excel2LaTeX from sheet 'Tabelle1'
\begin{table}
  \centering
  \caption{Messwerttabelle}
    \begin{tabular}{cc}
    T     & R \\
    -6.00 & 85.5 \\
    -5.75 & 84.8 \\
    -5.50 & 84.1 \\
    -5.25 & 83.3 \\
    -5.00 & 82.3 \\
    -4.75 & 81.1 \\
    -4.50 & 80.2 \\
    -4.25 & 79 \\
    -4.00 & 78.2 \\
    -3.75 & 77 \\
    -3.50 & 76.1 \\
    -3.25 & 75 \\
    -3.00 & 74.2 \\
    -2.75 & 73.6 \\
    -2.50 & 72.6 \\
    -2.25 & 71.8 \\
    -2.00 & 70.9 \\
    -1.75 & 70.1 \\
    -1.50 & 69.2 \\
    -1.25 & 68.8 \\
    -1.00 & 68.2 \\
    -0.75 & 67.5 \\
    -0.50 & 66.7 \\
    -0.25 & 66 \\
    0.00  & 65.1 \\
    \end{tabular}%
  \label{table:addlabel}%
\end{table}%
% Funktioniert, aber nicht schoen
Messwerttabelle \ref{tab:Tabellen:Messwerttabelle} mit \texttt{siunitx}.
\sisetup{%
		table-number-alignment=center,%
		table-column-width=0.18\textwidth-2\tabcolsep%
}%
\begin{longtable}{%
			S[table-figures-integer=1, table-figures-decimal=1]%
			S[table-figures-integer=4, table-figures-decimal=0,%
			table-sign-mantissa=true]%
			S[table-figures-integer=2, table-figures-decimal=0,%
			table-figures-exponent=1,table-sign-exponent,table-sign-mantissa=true]%
		}%
		\caption{Messwerttabelle}%
		\label{tab:Tabellen:Messwerttabelle}\\%
		\toprule
		{$I$} & {$R_\mathrm{z}$} & {$\Delta (R_\mathrm{z})_\mathrm{stat}$}\\
		{$\si{\milli\ampere}$} & {$\si{\per\second}$} &  {$\si{\per\second}$}  \\
		\midrule
		\endfirsthead
		\toprule
		{$I$} & {$R_\mathrm{z}$} & {$\Delta (R_\mathrm{z})_\mathrm{stat}$}\\
		{$\si{\milli\ampere}$} & {$\si{\per\second}$} &  {$\si{\per\second}$}  \\
		\midrule
		\endhead
		\bottomrule
		\endfoot
		\bottomrule
		\endlastfoot    
		0.1   & 524   	& 11e-6 	\\
		0.2   & -969   	& -22 		\\
		0.3   & -1374  	& -22e-4	\\
		0.4   & 1740  	& 30 
\end{longtable}

\chapter{Aufzählungen mit \texttt{enumerate, itemize}\texorpdfstring{, \url{https://github.com/labenbmic1}}{}}
\label{chap:Aufzaehlungen}
\begin{enumerate}
\item Ich bin Nummer 1
\item Ich bin Nummer 2
\item Was bin ich?
\end{enumerate}
Oder mit \texttt{itemize}:
\begin{itemize}
\item First
\item[H] Was ist hier falsch :D
\item Third
	\begin{enumerate}
	\item Yeah 1
	\item Yeah 2
	\end{enumerate}
\end{itemize}
\lipsum[1-5] Test, des \verb|\marginpar[links]{rechts}|\marginpar[Notiz (l)]{Notiz (r)} Randes. (links, wenn links Rand ist, ansonst wird der Text rechts ausgegeben), 
\lipsum[1-1] mit \verb|\marginline{}|\marginline{Test, \cite{LabenbacherTeX}, \ref{subfig:Gleitumgebungen:mit subcaptionbox:A}} erfolgt immer eine Ausgabe (egal ob links rechts). Besser: \verb|scrlayer-notecolumn|

\chapter{Richtiges Referenzieren mit \texttt{autoref, ref, footref}}
\label{chap:Richtiges Referenzieren}
\begin{table}\centering
	\caption{Referenzierungsmöglichkeiten}
	\label{tab:Referenzieren:Referenzierungsmoeglichkeiten}
	\begin{tabular}{c|c|c|c|c}%
& \TUMstyle{1}{autoref} & 
\TUMstyle{1}{ref} &
self-made-\TUMstyle{1}{ref} &
special \\
Part & 
\autoref{part:Gleitumgebungen in scrbook}	& 
\ref{part:Gleitumgebungen in scrbook} \\
Chapter & \autoref{chap:Einfuehrung:Hauptklassen}	 &
\ref{chap:Einfuehrung:Hauptklassen} \\
Section & \autoref{sec:Einfuehrung:Hauptklassen:Absatzauszeichnung} &
\ref{sec:Einfuehrung:Hauptklassen:Absatzauszeichnung} \\
Tabelle & \autoref{tab:Tabellen:A long table} & 
\ref{tab:Tabellen:A long table} \\
Tabelle & \autoref{subtab:Tabellen:mit latex-subtable:two} & 
\ref{subtab:Tabellen:mit latex-subtable:two} &
 & 
\subref{subtab:Tabellen:mit latex-subtable:two}\\
Abbildung & \autoref{fig:Gleitumgebungen:Grafik mit subcaptionbox ohne captionsetup (Standard: komafont)} & 
\ref{fig:Gleitumgebungen:Grafik mit subcaptionbox ohne captionsetup (Standard: komafont)} \\
Abbildung & \autoref{subfig:Gleitumgebungen:mit subcaptionbox:D} & 
\ref{subfig:Gleitumgebungen:mit subcaptionbox:D} &
& 
\subref{subfig:Gleitumgebungen:mit subcaptionbox:D}\\
Theorem & & \ref{theo:Theorem 1:theorem 1} \\
Lemma &  & \ref{lem:Theorem 1:lemma 1} \\
Korollar & & \ref{cor:Theorem 1:corollary 1} \\
Definition &  & \ref{def:Theorem 1:definition 1} \\
Remark & & \ref{rem:Theorem 1:remark 1} \\
Formel & & \eqref{equ:Mathematik:sub-gesamt} & & \eqref{subequ:Mathematik:sub-b} 
	\end{tabular}
\end{table}
Vom Buch: \cite{LabenbacherTeX}, oder siehe Fußnoten: Mit ref: \ref{footnote:Tabellen:A footnote in the tableentry-firsthead}, Mit \verb|\footref| (KomaScript): \footref{footnote:Tabellen:A footnote in the tableentry-firsthead} (Bei Fußnoten kein autoref, da dies nicht unterstützt wird!). 

\chapter{Mathematik mit/ohne \texttt{equbox}, und \texttt{chemformula}-Package}
\label{chap:Mathematik}
\begin{align}
\int\limits_{a}^{b} f \left( x \right) \dd{x} &= F(b) - F(a) \\
\pdv[n]{f}{x} &= \dfrac{x^{3}}{3} + \exp \left( -\lambda x \right) + \hypsine \left( x^{3} \right) + \exp \left( - \frac{\mathrm{i}}{x} \right) \notag\\
\sum\limits_{i=1}^{n} a_{n} &= e^{\mathrm{i} \pi } \cdot \sqrt[3]{n}  \xrightarrow{n \to \infty} \infty 
\end{align}
\begin{equbox}
	\begin{align}
	E=mc^2\label{equ:Mathematik:E = mc2}
	\end{align}%
\end{equbox}%
\begin{equbox}[Wenn es einen Titel bedarf. :D][breakable]
	\allowdisplaybreaks%
	\begin{align}%
	\int\nolimits_a^b \dd[3]{x} &= \int\limits_a^b \dd[3]{x}\\%
	\sum\nolimits_{n=1}^{\infty} &= \sum_{n=1}^\infty\\%
	\label{equ:Mathematik:Formel} \\%
	\ch{%
		RNO2 &<=>[ + e- ] RNO2^{-.} \\%
		RNO2^{-.} &<=>[ + e- ] RNO2^2-%
	} \\%
	a&=b
	\end{align}%
\end{equbox}%
Nun zum Paket siunitx: Man schreibt $a=\SI{5000}{\kilo\gram\metre\per\square\second}$ oder z.B. einfach mal etwas so wie $\si{\joule\per\mole\per\kelvin}$, $\si{\kilo\gram_{poly}\squared\per\mole_{cat}\per\hour}$ (für Hochzahlen muss cat als SI-Qualifier erklärt werden in der Preamble!), $\SI{3.00}{\MHz}$ \\%
\ch{
	!($1s^22s^1$)( "\chlewis{180.}{Li}" ) +%
	!($1s^22s^22p^5$)( "\chlewis{0.90,180,270}{F}" ) %
	-> %
	!($1s^2$)( Li+ ) + !($1s^22s^22p^6$)( "\chlewis{0,90,180,270}{F}" {}- )%
}%
\begin{subequations}
	\label{equ:Mathematik:sub-gesamt}
	\begin{eqnarray}
	{\cal M}=&ig_Z^2(4E_1E_2)^{1/2}(l_i^2)^{-1}
	(g_{\sigma_2}^e)^2\chi_{-\sigma_2}(p_2)\nonumber\\
	&\times
	[\epsilon_i]_{\sigma_1}\chi_{\sigma_1}(p_1)
	\label{subequ:Mathematik:sub-a}
	\end{eqnarray}
	\begin{equation}
	\left\{
	abc123456abcdef\alpha\beta\gamma\delta 1234556\alpha\beta
	\frac{1\sum^{a}_{b}}{A^2}
	\right\}
	\label{subequ:Mathematik:sub-b}
	\end{equation}
\end{subequations}
Die Formel \eqref{equ:Mathematik:sub-gesamt} besteht aus \eqref{subequ:Mathematik:sub-a} und \eqref{subequ:Mathematik:sub-b}.


\chapter{Theorem Teil 1 mit \texttt{tcolorbox}}%
\label{chap:Theorem 1}%

\lipsum[1-1]%
\begin{theorem}{Mittelwertsatz f\"{u}r $n$ Variable}{Theorem 1:Mittelwertsatz}%
	Es sei $n\in\mathbb{N}$, $D\subseteq\mathbb{R}^n$ eine offene Menge und $f\in C^{1}(D,\mathbb{R})$. Dann gibt es auf jeder Strecke $[x_0,x]\subset D$ einen Punkt $\xi\in[x_0,x]$, so dass gilt%
	\begin{equation*}%
	f(x)-f(x_0) = \operatorname{grad} f(\xi)^{\top}(x-x_0)%
	\end{equation*}%
\end{theorem}%
\begin{theorem*}{}{}%
	Ein Beispiel mit keiner Nummer.%
\end{theorem*}%
\begin{theorem}{}{}%
	Ein Beispiel mit keinem Titel.%
\end{theorem}%
\begin{lemma}{title :D}{Theorem 1:lemma 1}%
	Hallo Welt.
\end{lemma}%
\begin{corollary}{title :D}{Theorem 1:corollary 1}%
	Hallo Welt.
\end{corollary}%
\begin{proof}{}{}%
	\lipsum[1-3]
\end{proof}

\begin{definition}{title :D}{Theorem 1:definition 1}%
	\lipsum[1-1]
\end{definition}%
\begin{remark}{title :D}{Theorem 1:remark 1}%
	\lipsum[1-1]
\end{remark}%

\begin{theorem}{title :D}{Theorem 1:theorem 1}%
	\lipsum[1-1]
\end{theorem}%
\begin{proof*}{Titel?, zu Lemma \ref{lem:Theorem 1:lemma 1}}%
	\lipsum[1-1]
\end{proof*}

\chapter{Theorem Teil 2 mit \texttt{tcolorbox}}%
\label{chap:Theorem 2}%
\begin{theorem}{title :D}{Theorem 2:theorem 2}%
	\lipsum[1-1]
\end{theorem}%
Dies ist das Theorem \nameref{theo:Theorem 1:Mittelwertsatz}, oder Theorem \ref{theo:Theorem 1:Mittelwertsatz}. \lipsum[1-1]%
\newpage%
\layout%
\newpage%
\chapter{Indices mit \texttt{imakeidx} and Todos mit \texttt{todonotes}}%
\label{chap:IndicesTodos}%
%% Index-Optimierungen: for sorting numbers --> xindy instead of splitindex
%% Index-style-clist/tl for idxtext as 1 more optional parameter
A\index{A}A\index{Aa}A\index{Ab}A\index{Ac}A\index{Ad}
B\index{B}B\index{Ba}B\index{Bb}B\index{Bc}B\index{Bd}
C\index{C}C\index{Ca}C\index{Cb}C\index{Cc}C\index{Cd}
D\index{D}D\index{Da}D\index{Db}D\index{Dc}D\index{Dd}
E\index{E}E\index{Ea}E\index{Eb}E\index{Ec}E\index{Ed}
G\index{G}G\index{Ga}G\index{Gb}G\index{Gc}G\index{Gd}
G\index{G}G\index{Ga}G\index{Gb}G\index{Gc}G\index{Gd}
H\index{H}H\index{Ha}H\index{Hb}H\index{Hc}H\index{Hd}
I\index{I}I\index{Ia}I\index{Ib}I\index{Ic}I\index{Id}
J\index{J}J\index{Ja}J\index{Jb}J\index{Jc}J\index{Jd}
L\index{L}L\index{La}L\index{Lb}L\index{Lc}L\index{Ld}
K\index{K}K\index{Ka}K\index{Kb}K\index{Kc}K\index{Kd}
M\index{M}M\index{Ma}M\index{Mb}M\index{Mc}M\index{Md}
N\index{N}N\index{Na}N\index{Nb}N\index{Nc}N\index{Nd}
O\index{O}O\index{Oa}O\index{Ob}O\index{Oc}O\index{Od}
P\index{P}P\index{Pa}P\index{Pb}P\index{Pc}P\index{Pd}
Q\index{Q}Q\index{Qa}Q\index{Qb}Q\index{Qc}Q\index{Qd}
R\index{R}R\index{Ra}R\index{Rb}R\index{Rc}R\index{Rd}
S\index{S}S\index{Sa}S\index{Sb}S\index{Sc}S\index{Sd}
T\index{T}T\index{Ta}T\index{Tb}T\index{Tc}T\index{Td}
U\index{U}U\index{Ua}U\index{Ub}U\index{Uc}U\index{Ud}
V\index{V}V\index{Va}V\index{Vb}V\index{Vc}V\index{Vd}
W\index{W}W\index{Wa}W\index{Wb}W\index{Wc}W\index{Wd}
X\index{X}X\index{Xa}X\index{Xb}X\index{Xc}X\index{Xd}
Y\index{Y}Y\index{Ya}Y\index{Yb}Y\index{Yc}Y\index{Yd}
Z\index{Z}Z\index{Za}Z\index{Zb}Z\index{Zc}Z\index{Zd}
9\index{9}
\#\index{\#} \num{16.4}, \index{10}, \index{\#,A}, 
\index{\LaTeX}, % symbol
\index{LaTeX@\LaTeX}, % text
\index{9b}, \index{228},  \index{Papier!+@everystyle+@@\texttt{+@everystyle+@}|TUMstyle{idxnumber}}
\index{+@}, \index{+!}%

%\TUMstyle{number}{...} for emphasing text (or \TUMfont{number})
\TUMstyle{1}{Der} Mensch\index{Mensch} \TUMstyle{2}{ist} \TUMstyle{3}{ein} Tier\index{Tier}, das Geschäfte\index{Mensch!Geschäfte} macht; kein anderes Tier tut dies -- kein Hund\index{Tier!Hund} tauscht Knochen\index{Mensch!Knochen}\index{Tier!Knochen} mit einem anderen. \cite{LabenbacherTeX}. \index{Papier}, \index{Papier!Ausrichtung}, \index{Papier!Format}, \index{Papier!empty@\texttt{empty}}% 

L\\
La\todoMissing[caption={Missing.}]{I'm missing here\ldots.}\\
LaT\todoUnsure[caption={Unsure.}]{Unsure if that's good enough.}\index{LaTeX@\LaTeX}\\
LaTe\todoChange[caption={Change/Delete.}]{Please change this later on.}\index{LaTeX@\LaTeX!class|TUMstyle{idxnumber}}\\
LaTeX\todoInfo[caption={Info.}]{Nothing interesting.}\index{LaTeX@\LaTeX!package}\\
\LaTeX\todoImprovement[caption={Improvement.}]{Please Improve this code.}\index{LaTeX@\LaTeX!file}

\begin{longtable}{C{0.25\textwidth}Z{p}{0.2\textwidth}{\raggedleft}l}%
	\caption{A long table}\label{tab:introduction:1}\\%
	\toprule%
	f-head & f-head & f-head \\%
	\midrule%
	\endfirsthead%
	\toprule%
	head & head & head \\%
	\midrule%
	\endhead%
	\midrule%
	foot & foot & foot \\%
	\bottomrule%
	\endfoot%
	\midrule%
	l-foot & l-foot & l-foot \\%
	\bottomrule%
	\endlastfoot%
	%% =========================== %%
	a&b&c\\ a&b\todoChange[caption={Change/Delete.}]{Please delete this part of the introduction.}&c\\%
	%% =========================== %%
\end{longtable}%
\begin{figure}%
	%\includegraphicshelp% *[options]
	%\includegraphicshelp[]% *[options]
	%\includegraphicshelp[angle=45]% *[options]
	\todoincludegraphics{Missing number one.}% *[options]
	\todoincludegraphics[]{Missing number two.}% *[options]
	\todoincludegraphics[angle=90]{Missing number three.}% *[options]
	\caption[Test 1]{\lipsum[1]}%
	\label{fig:introduction:1}%
\end{figure}%
\lstinputlisting[style=python, caption={Python-Code für \ldots{}.}, label={lst:introduction:1}]{listings/test.py}%

\section{Sub-(ind/todo)}%
\label{sec:introduction:Sub-indtodo}%
\lipsum[1-1] Mensch\index{Mensch!Student}\newline%
\index{Mensch|(}%
\lipsum[1-3] Mensch\newline%
\lipsum[1-4] Mensch\todoImprovement[disable]{Everything that heightens the feeling of power in man, the will to power, power itself.}%
\lipsum[1-5] Mensch%\index{Mensch}\newline%
\lipsum[1-6] Mensch%\index{Mensch}\newline%
\lipsum[1-7] Mensch%\index{Mensch}\newline%
\lipsum[1-8] Mensch%\index{Mensch}\newline%
\lipsum[1-9] Mensch\index{Mensch|)}\newline%
\lipsum[1-10] Mensch\index{Mensch}\newline%
\lipsum[1-11] Mensch\index{Mensch}\newline%
\lipsum[1-12] Mensch\index{Mensch}\newline%
\lipsum[1-13] Mensch\index{Mensch|TUMstyle{idxnumber}}%
\lipsum[1-8]\index{Mensch|(}\lipsum[1-18] \index{Mensch|)}%

\begin{equation}
	a^2=b^2+c^2 \textrm{\todoChange{Delete this.}}
\end{equation}

%% end: LaTeX-Einfeuhrung.tex